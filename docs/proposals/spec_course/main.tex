%++++++++++++++++++++++++++++++++++++++++
% Don't modify this section unless you know what you're doing!
\documentclass[letterpaper,12pt]{article}
\usepackage{tabularx} % extra features for tabular environment
\usepackage{amsmath}  % improve math presentation
\usepackage{graphicx} % takes care of graphic including machinery
\usepackage[margin=1in,letterpaper]{geometry} % decreases margins
\usepackage{cite} % takes care of citations
\usepackage[final]{hyperref} % adds hyper links inside the generated pdf file
\hypersetup{
	colorlinks=true,       % false: boxed links; true: colored links
	linkcolor=blue,        % color of internal links
	citecolor=blue,        % color of links to bibliography
	filecolor=magenta,     % color of file links
	urlcolor=blue         
}
%++++++++++++++++++++++++++++++++++++++++


\begin{document}

\title{Specialization Course}
\author{Aakash S. Mallik, Prof. Di Wu, Prof. Yushan Pan}
\date{\today}
\maketitle

% \begin{abstract}
% The following proposal describes the 
% \textbf{Specialization Course (IE505818)} structure that would be required for the topic \textbf{Affordance Learning for Interaction Design}.
% \end{abstract}


\section{Course code and Name}

IE505818 Specialization Course

\section{Topic}

Affordance Learning for Interaction Design

\section{Course content}

\begin{enumerate}
	\item Affordance Theory
	\begin{enumerate}
		\item Basic concepts in affordance theory.
        \item Affordance analysis methods.
  
        \item Applications of Affordance Theory in the field of Interaction Design.
		\item Recent advances in Affordance Theory.
	\end{enumerate}
	\item Reinforcement Learning
	\begin{enumerate}
		\item Reinforcement Learning basics.
        \item Evaluation methods for Reinforcement Learning algorithms.
		\item Reinforcement Learning applications in the interaction design.
	\end{enumerate}
\end{enumerate}

\section{Learning outcome}

\begin{enumerate}
	\item Knowledge
	\begin{enumerate}
		\item To understand Affordance Theory in the context of Interaction Design.
		\item To comprehend and compare the different Reinforcement Learning algorithms.
		\item To connect the research of affordance theory with applications in robotic design.
	\end{enumerate}
	\item Skill
	\begin{enumerate}
		\item Literature review in the related research fields.
		\item Technical skills to implement different RL algorithms and techniques.
  
	\end{enumerate}
	\item General Competence
	\begin{enumerate}
		\item Ability to communicate affordance theory and deep learning concepts and main challenges in the related areas.
	\end{enumerate}
\end{enumerate}

\section{Evaluation}

\begin{enumerate}
	\item Oral Exam, including a presentation about topic introduction, related works, problem analysis, evaluation and conclusions.
	
\end{enumerate}

\section{References}

\begin{enumerate}
	\item The Design of Everyday Things by Donald A. Norman
	\item Affordance, Conventions, and Design by Donald A. Norman
	\item Rediscovering Affordance:
	A Reinforcement Learning Perspective by Todi et al.
	\item Affordance Learning for End-to-End Visuomotor Robot Control by Levine et al.
	\item Technology Affordances by William W. Gaver
\end{enumerate}

\end{document}
